\cleardoublepage % memastikan bab baru dimulai di halaman ganjil (kanan)
\chapter{STUDI LITERATUR}
\label{chap:studi-literatur}
\section{Tentang Studi Literatur}
Studi literatur biasanya berisi tentang tinjauan pustaka, landasan teori, dan penelitian-penelitian terdahulu yang relevan dengan topik tugas akhir yang dikerjakan.
Dalam subbab ini, penulis diharapkan dapat menjelaskan:
\begin{enumerate}
  \item Landasan teori yang diperoleh dari literatur yang akan digunakan untuk menyelesaikan persoalan TA;
  \item Pengetahuan tentang kasus yang akan dikaji; dan
  \item Penelitian atau solusi terkait untuk mengetahui posisi persoalan dan berbagai solusi yang memungkinkan untuk diterapkan berdasarkan hasil studi literatur ini.
\end{enumerate}
Jadi, studi literatur ini bukan sekadar rangkuman dari berbagai sumber referensi, tetapi harus diolah dan disajikan secara sistematis sehingga dapat memberikan gambaran yang jelas tentang landasan teori dan penelitian-penelitian terdahulu yang relevan dengan topik tugas akhir.

Penulisan studi literatur yang baik dimulai dari penjelasan tentang kasus yang akan dikaji, kemudian diikuti dengan penjelasan tentang teori-teori yang relevan, dan diakhiri dengan tinjauan terhadap penelitian-penelitian terdahulu yang berkaitan dengan topik tugas akhir.
Misalnya, jika topik tugas akhir adalah tentang masalah klasifikasi biji kopi berdasarkan citra kopi, 
penulis dapat memulai dengan menjelaskan terlebih dahulu tentang karakteristik biji kopi, masalah dalam melakukan klasifikasi biji kopi, 
kemudian menjelaskan tentang berbagai metode berbasis komputer yang dapat digunakan untuk mengklasifikasikan biji kopi berdasarkan citranya, 
metode terbaik yang memungkinkan untuk diterapkan, teori-teori dasar tentang metode tersebut, dan diakhiri dengan tinjauan terhadap penelitian-penelitian terdahulu yang menggunakan metode untuk klasifikasi citra.

Dalam studi literatur, biasanya banyak penjelasan tentang teori-teori yang disertai dengan rumus atau persamaan matematika, gambar, tabel, algoritma, pseudocode, atau kode program.
Oleh karena itu, pada subbab berikut akan dijelaskan beberapa hal teknis terkait format penulisan dokumen tugas akhir.

\section{Format Penulisan Gambar, Tabel, Rumus, dan Kode Program}
Berikut adalah beberapa contoh penulisan gambar, tabel, rumus, dan kode program yang sesuai dengan format penulisan tugas akhir.
\subsection{Penulisan Gambar}
Contoh gambar dapat dilihat pada Gambar \ref{fig:jaringan} dan Gambar \ref{fig:jaringan2}.
Secara vertikal, gambar biasanya diletakkan di bagian atas halaman (\textit{top}) atau di bagian bawah halaman (\textit{bottom}).
Gambar tidak harus diletakkan tepat di tempat gambar tersebut disebutkan pertama kali dalam teks,
tetapi dapat diletakkan di halaman berikutnya jika tidak muat di halaman yang sama. Pada contoh ini, kedua gambar diletakkan di bagian atas halaman 
menggunakan opsi \texttt{[t]} pada lingkungan \texttt{figure}, namun pada halaman yang berbeda, meskipun disebutkan di halaman yang sama akibat ruang yang terbatas. 

Penyebutan judul gambar dalam teks menggunakan perintah \texttt{ref}, misalnya Gambar \texttt{\\ref\{fig:jaringan\}}.
Gunakan huruf kapital pada kata "Gambar" saat menyebutkan gambar dalam teks.

\begin{figure}[t] % pilihan opsi yang disarankan: t = top, b = bottom, h = here
  \caption{Contoh peletakan gambar di bagian atas halaman} 
	\label{fig:jaringan}
    	\includegraphics[width=0.7\textwidth]{images/gambar1.png}
\end{figure}

Gambar, grafik, atau ilustrasi harus memiliki keterangan atau judul (\textit{caption}) yang menjelaskan isi gambar tersebut. 
Secara horisontal, gambar dan judulnya diposisikan di tengah. Nomor gambar tidak diakhiri tanda baca.
Judul gambar diletakkan di bawah gambar dan ditulis dengan huruf kecil kecuali huruf pertama pada kalimat judul. 
Kalimat judul yang panjang dapat ditulis dalam beberapa baris seperti pada Gambar \ref{fig:jaringan2}.

Ukuran gambar yang ditampilkan dapat diatur dengan mengubah nilai \textit{width} dalam sintaks \textit{includegraphics}. 
Gambar \ref{fig:jaringan} memiliki lebar 70\% dari lebar teks, sedangkan Gambar \ref{fig:jaringan2} memiliki lebar 90\% dari lebar teks.
 

\begin{figure}[t] 
  \caption{Contoh peletakan gambar dengan judul yang panjang sehingga ditulis dalam beberapa baris} 
	\label{fig:jaringan2}
    	\includegraphics[width=0.9\textwidth]{images/gambar1.png}
\end{figure}

Pastikan gambar yang digunakan memiliki resolusi yang cukup tinggi agar tidak pecah atau buram saat dicetak. 
Gambar umumnya tidak tajam dan tulisan tidak jelas atau kabur jika gambar tersebut:
\begin{enumerate}[a.]
  \item diperoleh dari hasil cropping pada suatu halaman buku atau situs web;
  \item hasil pembesaran gambar yang gambar aslinya sebenarnya berukuran kecil; atau
  \item disimpan dalam resolusi kecil
\end{enumerate}
Ketidakjelasan gambar ini dapat dilihat pada garis-garis diagram yang tidak tegas dan tulisan-tulisan dalam gambar yang tampak kabur dan kurang jelas terbaca.

Untuk mendapatkan gambar yang tidak kabur (\textit{blur}), langkah-langkah berikut dapat digunakan:
\begin{enumerate}[(a)]
\item Hindari \textit{screenshot}. Gambar yang didapat di suatu pustaka atau referensi sebaiknya digambar ulang (tidak perlu sama persis), misalnya menggunakan PowerPoint, Canva, Figma, draw.io, atau yang lainnya.
\item Jika diagram atau ilustrasi digambar menggunakan draw.io, saat gambar disimpan ke format PNG atau JPG (\textit{export as}), lakukan \textit{zoom} ke minimal 300\% (\textit{the default value is} 100\%). 
\item Jika diagram digambar dengan menggunakan \texttt{PowerPoint} atau aplikasi yang mengolah gambar vektor lainnya, gambar dapat langsung disimpan asalkan resolusinya tinggi.
\end{enumerate}

\subsection{Penulisan Tabel}
Format penulisan tabel hampir sama dengan format penulisan gambar. Bedanya adalah judul tabel diletakkan di atas tabel, bukan di bawah tabel seperti pada gambar.
Ada tabel yang bisa dimuat dalam satu halaman (tabel pendek) dan ada tabel panjang yang yang tidak muat dalam satu halaman. Kedua jenis tabel tersebut disusun dengan cara yang berbeda.
Jika tabel sangat panjang, gunakan paket \textit{longtable} untuk membuat tabel yang dapat terpotong ke halaman berikutnya.
Namun, jika tabel dapat dimuat dalam satu halaman, gunakan lingkungan \textit{table} biasa, tidak menggunakan \textit{longtable}.
Usahakan tabel dapat ditulis dalam satu halaman, tidak terpotong ke halaman berikutnya, jika memungkinkan.

\subsubsection{Penulisan Tabel Pendek}
Contoh tabel pendek dapat dilihat pada Tabel \ref{tab:tabel1}, \ref{tab:tabel2}, dan \ref{tab:tabel3}. 
Pada kedua table ini, tabel dan judulnya diposisikan di tengah secara horisontal dan 
judul tabel diletakkan di atas tabel. Secara vertikal, tabel sebaiknya diletakkan di bagian atas atau bawah halaman.

Penyebutan judul tabel dalam teks menggunakan perintah \texttt{ref}, misalnya Tabel \ref{tab:tabel1}.
Gunakan huruf kapital pada kata "Tabel" saat menyebutkan tabel dalam teks.

Tabel \ref{tab:tabel1} menggunakan paket \textit{tabularx} untuk mengatur lebar kolom secara fleksibel. 
Selain itu, tabel ini juga menggunakan \textit{threeparttable} untuk menambahkan catatan kaki pada tabel 
dan opsi \texttt{X} agar lebar kolom fleksibel menyesuaikan lebar tabel.
Catatan kaki pada tabel ini menggunakan perintah \texttt{tablenotes} yang disediakan oleh paket \textit{threeparttable}.
Catatan kaki pada tabel ini menggunakan perintah \texttt{\textbackslash footnotemark} dan \texttt{\textbackslash footnotetext} untuk menambahkan catatan kaki pada tabel.

Tabel \ref{tab:tabel2} menggunakan format standar \texttt{| l | c | r |}. 
Tabel \ref{tab:tabel3} menggunakan lebar kolom tetap (\textit{fixed width}) menggunakan \texttt{p\{width\}}.

\begin{table} 
  \centering
    \begin{threeparttable}[t] % threeparttable agar bisa menambahkan catatan kaki pada tabel
    \caption{Contoh tabel yang menggunakan \texttt{tabularx} untuk mengatur lebar kolom secara fleksibel. 
    Selain itu, tabel ini juga menggunakan \texttt{threeparttable} untuk menambahkan catatan kaki pada tabel 
    dan opsi \texttt{X} agar lebar kolom fleksibel menyesuaikan lebar tabel.}
    \label{tab:tabel1}
    \begin{tabularx}{0.6\textwidth}{| c | X| l | r @{}|} % @{} removes extra space at edges, X makes the column flexible width
	    \hline
	    No & Nama 	& Satuan 		& Harga \\
	    \hline
	    1 & Buku 	& Exemplar	& 25000 \\
	    2 & Komputer	& Unit		& 2500000 \\
	    3 & Pensil	& Buah		& 118900 \\
	    \hline
	  \end{tabularx} 
    \begin{tablenotes}
      \footnotesize
      \item \footnotemark[1]Tabel harga berdasarkan data tahun 2023.
      \item \footnotemark[2]Sumber: \url{https://www.example.com}
    \end{tablenotes}
  \end{threeparttable}
\end{table}




\begin{table}[t] % pilihan opsi yang disarankan: t = top, b = bottom, h = here
  \caption{Tabel yang menggunakan format standar \texttt{| l | c | r |}}
  \label{tab:tabel2}
	\begin{tabular}{ | l | c | r | }
	\hline
	Nama 	& Satuan 		& Harga \\
	\hline
	Buku 	& Exemplar	& 25000 \\
	Komputer	& Unit		& 2500000 \\
	Pensil	& Buah		& 118900 \\
	\hline
	\end{tabular}
\end{table}


\begin{table}[t] % pilihan opsi yang disarankan: t = top, b = bottom, h = here
  \centering
  \caption{Tabel yang lebar kolomnya dibuat tetap}
  \label{tbl:tabel3}
	\begin{tabular}{  p{0.2\textwidth} | p{0.2\textwidth} | p{0.2\textwidth}  }
	\hline
	\textbf{Nama} 	& \textbf{Satuan} 		& \textbf{Harga} \\
	\hline
	Buku 	& Exemplar	& 25000 \\
	\emph{Komputer}	& Unit		& 2500000 \\
	Pensil	& Buah		& 118900 \\
	\hline
	\end{tabular}
\end{table}

% -- Example of importing table from external file --
\subsubsection{Mengimpor Tabel dari Berkas Eksternal}

Untuk kerapian penulisan, tabel tidak harus ditulis di dokumen utama. Sebagai contoh, Tabel \ref{tab:tabel4} diimpor dari berkas eksternal \texttt{table/tabel1.tex} menggunakan perintah \texttt{input}. 
Dengan demikian, jika tabel tersebut perlu diubah, cukup mengubah pada berkas eksternal tersebut tanpa perlu mengubah pada berkas dokumen utama ini.

\input tables/tabel1.tex


% -- Example of long table --
\subsubsection{Tabel yang Sangat Panjang}
Jika tabel terlalu panjang sehingga tidak muat dalam satu halaman, gunakan paket 
\textit{longtable} untuk membuat tabel yang dapat terpotong ke halaman berikutnya, 
seperti pada Tabel \ref{tab:long}. 

\input tables/longtable1.tex

\subsection{Penulisan Rumus atau Persamaan Matematika Menggunakan {\LaTeX} dan Penomorannya}
Contoh rumus matematika dapat ditulis seperti pada Persamaan \ref{eq:contoh1} di bawah ini. 
Penomoran persamaan diletakkan di sebelah kanan, dan rumus ditulis dalam mode \textit{display math}.
\begin{equation}
E = mc^2
\label{eq:contoh1}
\end{equation}

Contoh lain penulisan rumus matematika yang lebih kompleks dapat ditulis seperti pada Persamaan \ref{eq:rumus2}.

\begin{align}
f(x) &= ax^2 + bx + c \\
f'(x) &= \frac{d}{dx}(ax^2 + bx + c) \notag \\ % tidak menampilkan nomor pada baris ini
      &= 2ax + b \label{eq:rumus2}
\end{align}

Jika rumus terlalu panjang untuk ditulis dalam satu baris, gunakan lingkungan \textit{multline} seperti pada Persamaan \ref{eq:rumus3} di bawah ini.
\begin{multline} 
y = a_0 + a_1x + a_2x^2 + a_3x^3 + a_4x^4 + a_5x^5 + a_6x^6 + a_7x^7 \\
    + a_8x^8 + a_9x^9 + a_{10}x^{10} \label{eq:rumus3}
\end{multline}

Jika ada penurunan rumus yang terdiri dari beberapa baris, namun tidak memerlukan penomoran pada setiap baris, gunakan lingkungan \textit{align*}, misalnya:

\begin{align*} 
S &= \sum_{i=1}^{n} i^2 \\
  &= 1^2 + 2^2 + 3^2 + \cdots + n^2 \\
  &= \frac{n(n + 1)(2n + 1)}{6}
\intertext{Contoh lainnya adalah rumus untuk mencari nilai rata-rata fungsi $f(x)$ pada interval $[p, q]$:}
\bar{f} &= \frac{1}{q - p} \int_{p}^{q} f(x) \, dx \\
        &= \frac{1}{q - p} \int_{p}^{q} (ax^2 + bx + c) \, dx \\
        &= \frac{1}{q - p} \left[ \frac{a}{3}x^3 + \frac{b}{2}x^2 + cx \right]_p^q \\
        &= \frac{a(q^3 - p^3)}{3(q - p)} + \frac{b(q^2 - p^2)}{2(q - p)} + c \label{eq:rumus4}
\end{align*}



\subsection{Penulisan Kode Program, \textit{Script} atau \textit{Pseudocode}}
Contoh penulisan kode program, \textit{script}, atau \textit{pseudocode} dapat dilihat seperti pada \textit{listing} \ref{lst:binary_iter}. 
Gunakan paket \textit{listings} untuk menulis \textit{source code} dalam bahasa pemrograman tertentu. 
Kode seperti ini disebut dengan istilah \textit{listing}. 
Kode program, \textit{script}, atau \textit{pseudocode} yang relatif pendek dapat ditulis di dalam
dokumen utama. Namun, jika kode programnya sangat panjang, sebaiknya ditulis di berkas terpisah dan diletakkan di lampiran.

\subsection{Penulisan Algoritma}
Contoh penulisan algoritma dapat dilihat pada Algoritma \ref{alg:binary-search}. Ada beberapa paket yang dapat digunakan untuk menulis algoritma,
misalnya \textit{algorithm2e}, \textit{algorithmic}, dan \textit{algpseudocode}. 
Pada contoh ini digunakan paket \textit{algorithmic} untuk menulis algoritma. Setiap paket memiliki sintaks yang berbeda-beda.
Oleh karena itu, pelajari dokumentasi paket yang digunakan untuk menulis algoritma.


\input{listings/binary-search-python.tex}


\input{algorithms/binary-search-alg.tex}



