%! TEX program = xelatex
% 
% Versi 1.1
%
% Template Tugas Akhir 
% Program Studi Sistem dan Teknologi Informasi
% Sekolah Teknik Elektro dan Informatika
% Institut Teknologi Bandung
% 
% Dibuat oleh: IGB Baskara Nugraha 
% Email: baskara@itb.ac.id 
% 
%
% Petunjuk penggunaan:
% 1. Ada 2 file utama, yaitu TA.tex (file ini) dan daftar-pustaka.bib (file daftar pustaka).
% 2. Sunting TA.tex sesuai dengan kebutuhan Anda.
% 3. Sunting atau generate isi daftar-pustaka.bib dengan referensi yang Anda gunakan, sesuai dengan format BibLaTeX.
% 4. Simpan kedua file tersebut dalam satu folder yang sama.
% 5. Kompilasi file TA.tex menggunakan XeLaTeX dan Biber (lihat urutan cara kompilasi di bawah).
% 6. Hasil kompilasi adalah file TA.pdf yang siap dicetak.
% 
% Urutan cara kompilasi (melalui command line):
% 1. xelatex TA.tex
% 2. biber TA      
% 3. xelatex  TA.tex
% 4. xelatex  TA.tex
%
% Catatan:
% - Pastikan Anda telah menginstal paket-paket LaTeX yang diperlukan, termasuk
%   biblatex-chicago dan fontspec.
% - Gunakan editor LaTeX yang mendukung XeLaTeX, seperti TeXstudio, Overleaf, atau lainnya.
% - Jika meenggunakan Visual Studio Code sebagai editor, pastikan mengatur "latex-workshop.latex.tools" dan
%   "latex-workshop.latex.recipes" untuk mendukung XeLaTeX dan Biber dengan cara menambahkan konfigurasi berikut:
%   "latex-workshop.latex.tools": [ 
%       {
%           "name": "xelatex",
%           "command": "xelatex",
%           "args": [
%               "-synctex=1",
%               "-interaction=nonstopmode",
%               "-file-line-error",
%               "%DOC%"
%           ]
%       },
%       {
%           "name": "biber",
%           "command": "biber",
%           "args": [
%               "%DOCFILE%"
%           ]
%       }
%   ],
%   "latex-workshop.latex.recipes": [
%       {
%           "name": "xelatex -> biber -> xelatex*2",
%           "tools": [
%               "xelatex",
%               "biber",
%               "xelatex",
%               "xelatex"
%           ]
%       }
%   ]
% - Untuk referensi lebih lanjut tentang penggunaan BibLaTeX dengan gaya Chicago, silakan merujuk ke dokumentasi resmi BibLaTeX.
%   https://ctan.org/pkg/biblatex-chicago
% - Untuk referensi lebih lanjut tentang penggunaan XeLaTeX dan fontspec, silakan merujuk ke dokumentasi resmi fontspec.
%   https://ctan.org/pkg/fontspec
% - Selamat menyusun dokumen tugas akhir Anda!
%
\documentclass[12pt,a4paper]{book}

% --- Main packages ---
\usepackage[utf8]{inputenc} % for UTF-8 encoding
\usepackage{fontspec} % for font selection
\usepackage[indonesian]{babel} % untuk bahasa Indonesia
\usepackage{csquotes} % for context-sensitive quotation facilities
\usepackage{setspace} % for line spacing
\usepackage{graphicx} % for images
\usepackage{caption} % for customizing captions
\usepackage{subcaption} % for sub-figures
\usepackage{hyperref} % for hyperlinks
\usepackage{tocloft} % for customizing table of contents
\usepackage{titlesec} % for customizing titles
\usepackage{lipsum} % for dummy text (lorem ipsum text)
\usepackage{floatrow} % for customizing float (figure and table) positions
\usepackage{listings} % for code listing
\usepackage{amsmath} % for math
\usepackage{amssymb} % for math symbols
\usepackage[shortlabels]{enumitem} % for customizing lists
\usepackage[skip=12pt]{parskip} % for spacing between paragraphs
\usepackage{longtable} % for long tables
\usepackage{booktabs} % for better table rules
\usepackage[bahasai]{datetime2} % for date formatting in Indonesian
\usepackage{geometry} % for page margins
\usepackage{algorithm} % for algorithms
\usepackage{algorithmicx} % for algorithmic environment
\usepackage{algpseudocode} % for pseudocode
\usepackage{microtype} % Untuk optimasi typography dan spacing
\usepackage{floatrow} % for customizing float (figure and table) positions
\usepackage{array} % for better arrays (eg. in tabular environment)
\usepackage{threeparttable} % for tables with notes
\usepackage{tabularx} % for tables with adjustable-width columns
\usepackage{xcolor} % for color definitions
\usepackage[title,titletoc]{appendix} % for appendix management
\usepackage{csquotes} % for context-sensitive quotation facilities

\setmainfont{Times New Roman} % set main font to Times New Roman
\onehalfspacing % spasi 1.5

% --- Bibliography menggunakan BibLaTeX dengan gaya Chicago ---
\usepackage[
    backend=biber,
    authordate,
    language=english,
    autolang=other
]{biblatex-chicago}
\addbibresource{daftar-pustaka.bib}

% --- Terjemahan istilah daftar pustaka ke Bahasa Indonesia ---
\DefineBibliographyStrings{english}{
  and          = {dan},
  andothers    = {dkk.},
  editor       = {penyunting},
  editors      = {penyunting},
  translator   = {penerjemah},
  byeditor     = {disunting oleh},
  bytranslator = {diterjemahkan oleh},
  in           = {dalam},
  edition      = {edisi},
  pages        = {hal.},
  page         = {hal.},
  volume       = {vol.},
  number       = {no.},
  urlseen      = {diakses pada},
  url          = {tautan},
}

% --- ubah format sitasi menjadi (Author Year) ---
\let\oldcite\cite
\renewcommand{\cite}{\parencite}

% --- hilangkan koma sebelum 'dan' pada daftar pustaka ---
\renewcommand*{\finalandcomma}{} 

% --- hyperlinks setup ---
\hypersetup{
    colorlinks=true,
    linkcolor=black,
    citecolor=black,
    urlcolor=black
}

% -- No Header dan No Footer ---
\pagestyle{plain}

% --- Ubah nama list of listing ke "DAFTAR LISTING" ---
\renewcommand{\lstlistlistingname}{DAFTAR \textit{LISTING}} 
\renewcommand{\lstlistingname}{\textit{Listing}}
\lstset{basicstyle=\ttfamily\footnotesize,breaklines=true}

% --- Ubah nama list of algorithm ke "DAFTAR ALGORITMA" ---
\renewcommand{\algorithmname}{Algoritma}
\renewcommand{\listalgorithmname}{DAFTAR ALGORITMA}
\lstset{basicstyle=\ttfamily\footnotesize,breaklines=true}
\renewcommand{\thealgorithm}{\thechapter.\arabic{algorithm}}

\renewcommand \cftchapdotsep{4.5} % Atur jarak titik-titik pada daftar isi

%\cftsetrmarg{4em}  % agar judul subbab tidak mepet ke nomor halaman di daftar isi (optional)
% -- Atur indentasi judul bab, subbab, dan subsubbab di daftar isi ---
\setlength{\cftchapindent}{0em}
\setlength{\cftchapnumwidth}{1.8em}
\setlength{\cftsecindent}{1.8em}
\setlength{\cftsecnumwidth}{2.4em}
\setlength{\cftsubsecindent}{4.2em}
\setlength{\cftsubsecnumwidth}{2.8em}
\setlength{\cftsubsubsecindent}{7.0em}
\setlength{\cftsubsubsecnumwidth}{3.2em}

% --- Ubah nama bulan ke Bahasa Indonesia ---
\renewcommand*{\DTMbahasaimonthname}[1]{%
  \ifcase#1 %
    \or Januari%  % 1st month
    \or Februari%  % 2nd month
    \or Maret%  % 3rd month
    \or April%  % 4th month
    \or Mei%  % 5th month
    \or Juni%  % 6th month
    \or Juli%  % 7th month
    \or Agustus%  % 8th month
    \or September%  % 9th month
    \or Oktober%  % 10th month
    \or November%  % 11th month
    \or Desember%  % 12th month
  \fi
}

% --- Atur margin halaman untuk single-sided format (satu muka) ---
\geometry{
      a4paper,
      inner=4cm,
      outer=3cm,
      top=3cm,
      bottom=3cm
}

% --- LISTINGS STYLE FOR PYTHON CODE ---
% --- Silakan diubah sesuai selera dan kebutuhan ---
% Define colors for syntax highlighting
\definecolor{codegreen}{rgb}{0,0.6,0}
\definecolor{codegray}{rgb}{0.5,0.5,0.5}
\definecolor{codepurple}{rgb}{0.58,0,0.82}
\definecolor{backcolour}{rgb}{0.95,0.95,0.92}

% Configure listings style
\lstdefinestyle{pythonstyle}{
    backgroundcolor=\color{backcolour},   
    commentstyle=\color{codegreen},
    keywordstyle=\color{magenta},
    numberstyle=\tiny\color{codegray},
    stringstyle=\color{codepurple},
    basicstyle=\ttfamily\footnotesize,
    breakatwhitespace=false,         
    breaklines=true,                 
    captionpos=b,                    
    keepspaces=true,                 
    numbers=left,                    
    numbersep=5pt,                  
    showspaces=false,                
    showstringspaces=false,
    showtabs=false,                  
    tabsize=4,
    language=Python
}
\lstset{style=pythonstyle}
% --- END OF LISTINGS STYLE FOR PYTHON CODE ---

% --- FORMAT TAMPILAN JUDUL BAB, SUBBAB, JUDUL GAMBAR DAN TABEL ---
\setcounter{tocdepth}{4} % kedalaman daftar isi sampai subsubbab
\setcounter{secnumdepth}{4} % kedalaman penomoran sampai subsubbab
% --- Judul Bab ---
\titleformat{\chapter}[display]
      {\centering\normalfont\large\bfseries} % Commands for the entire chapter title
      {\MakeUppercase \chaptertitlename\ \thechapter}{0pt}{\large} % Chapter number format
      \renewcommand\thechapter{\Roman{chapter}}
% --- Atur spasi sebelum dan sesudah judul bab/chapter ---  
\titlespacing{\chapter}{0pt}{0pt}{5pt}
\titlespacing{\section}{0pt}{6pt}{6pt}
\titlespacing{\subsection}{0pt}{6pt}{6pt}
\titlespacing{\subsubsection}{0pt}{6pt}{6pt}

% --- Judul Subbab dan Subsubbab ---
\titleformat{\section}
	{\normalfont\bfseries}
	{\thesection}{1em}{}
\titleformat{\subsection}
	{\normalfont\bfseries}
	{\thesubsection}{1em}{}
\titleformat{\subsubsection}
	{\normalfont\bfseries}
	{\thesubsubsection}{1em}{}

% --- Gambar ---
\captionsetup[figure]{
  labelsep=space, 
  format=hang
}  
\floatsetup[figure]{
  capposition=bottom
} 
% --- Tabel ---
\captionsetup[table]{
  labelsep=space, 
  format=hang
} 
\floatsetup[table]{
  capposition=top, 
  objectset=centering, 
  captionskip=6pt
} 
% --- Listing ---
\captionsetup[lstlisting]{
  labelsep=space, 
  format=hang, 
  justification=raggedright, 
  singlelinecheck=false
}
\floatsetup[lstlisting]{
  capposition=top
}
% --- Algoritma ---
\captionsetup[algorithm]{
  labelsep=space, 
  format=hang, 
  justification=raggedright, 
  singlelinecheck=false
}
\floatsetup[algorithm]{
  capposition=top
}

% --- Atur indentasi paragraf ---
\setlength{\parindent}{0pt}
\setlength{\parskip}{12pt plus 2pt minus 1pt}

\setlist[enumerate]{nosep, topsep=-10pt} % mengurangi spasi antar item dan atas bawah daftar
\setlist[itemize]{nosep, topsep=-10pt} % mengurangi spasi antar item dan atas bawah daftar

% ==========================================
% AWAL DOKUMEN
% ==========================================
\begin{document}

% ==========================================
% HALAMAN AWAL (FRONTMATTER)
% ==========================================
\frontmatter
% --- Atur format daftar isi untuk bagian frontmatter ---
\addtocontents{toc}{\protect\setlength{\cftbeforechapskip}{0pt}}
\addtocontents{toc}{\protect\renewcommand{\protect\cftchapfont}{\protect\normalfont}}
\addtocontents{toc}{\protect\renewcommand{\protect\cftchappagefont}{\protect\normalfont}}
\renewcommand{\cftchapleader}{\cftdotfill{\cftchapdotsep}}
% ==========================================

\input{1 Halaman Judul.tex}
\input{2 Lembar Pengesahan.tex}
\chapter*{LEMBAR PERNYATAAN}
\addcontentsline{toc}{chapter}{LEMBAR PERNYATAAN}
\vspace{2cm}
Saya yang bertanda tangan di bawah ini menyatakan dengan sesungguhnya bahwa:
\begin{enumerate}
    \item Tugas Akhir ini adalah hasil karya saya sendiri yang dibuat dengan sebenar-benarnya sesuai dengan kaidah ilmiah dan etika akademik.
    \item Semua sumber rujukan dan data yang saya gunakan dalam penyusunan laporan tugas akhir ini, baik yang berupa kutipan langsung maupun tidak langsung, telah saya cantumkan sumbernya dengan baik dan benar sesuai dengan ketentuan penulisan laporan tugas akhir.
    \item Isi laporan ini belum pernah diajukan pada program pendidikan di institusi mana pun.
\end{enumerate}
Apabila di kemudian hari terbukti bahwa laporan ini melanggar hal-hal di atas, saya bersedia menerima sanksi sesuai dengan peraturan yang berlaku di Institut Teknologi Bandung.
\\[2cm]
Bandung, \today\\
\\[2cm]

\underline{John Doe} \\
NIM: 18299000
  
\input{4 Abstrak.tex}
\input{5 Kata Pengantar.tex}
% ==========================================
% --- Atur margin halaman untuk double-sided format (bolak-balik) ---
% --- Comment baris di bawah ini jika ingin tetap menggunakan single-sided format ---
% --- atau pindahkan ke bagian setelah \mainmatter jika hanya ingin halaman utama saja yang bolak-balik ---

\newgeometry{
      twoside,
      a4paper,
      inner=4cm,
      outer=3cm,
      top=3cm,
      bottom=3cm
}
% ==========================================


\cleardoublepage
\addcontentsline{toc}{chapter}{DAFTAR ISI}
\vspace{2cm}
\begin{center}
      {\fontsize{14}{16.8}\selectfont\textbf{DAFTAR ISI}}\\[1em]
\end{center}
\vspace{1em}

{\singlespacing\setlength{\parskip}{0pt}
      \makeatletter
     \@starttoc{toc}
      \makeatother
}
%\input{7 Daftar Lampiran.tex} % Optional. Daftar lampiran sudah ada di dalam daftar isi.
\input{8 Daftar Gambar.tex}
\input{9 Daftar Tabel.tex}
\input{10 Daftar Algoritma.tex}
\cleardoublepage
\titlespacing*{\chapter}{0pt}{40pt}{20pt}
\lstlistoflistings
\addcontentsline{toc}{chapter}{DAFTAR KODE PROGRAM}
\label{chap:daftar-listing}
\setcounter{chapter}{0}
\vspace{2cm}
\begin{center}
{\fontsize{14}{16.8}\selectfont\textbf{DAFTAR SINGKATAN}}\\[1em]
\end{center}
\label{chap:singkatan}
\addcontentsline{toc}{chapter}{DAFTAR SINGKATAN}
\setcounter{chapter}{0}
\setcounter{section}{0}
\vspace{2em}




\begin{longtable}{@{} l >{\raggedright\arraybackslash}p{7.5cm} c @{}}
\multicolumn{1}{l}{\textbf{Singkatan}} & \multicolumn{1}{l}{\textbf{Deskripsi}} & \multicolumn{1}{l}{\parbox[c]{3.5cm}{\raggedright \textbf{Pemakaian pertama kali}}} \\

% ============================================================
% DATA SINGKATAN (URUT ABJAD A-Z)
% ============================================================

AI & \textit{Artificial Intelligence} & 1 \\
CNN & \textit{Convolutional Neural Network} & 2 \\
GPU & \textit{Graphics Processing Unit} & 14 \\


\end{longtable}


\input{13 Daftar Simbol.tex}

% ==========================================
% BAGIAN UTAMA DOKUMEN
% ==========================================
\mainmatter
% --- Atur format daftar isi untuk bagian mainmatter ---
\addtocontents{toc}{\protect\setlength{\cftbeforechapskip}{10pt}}
\addtocontents{toc}{\protect\renewcommand{\protect\cftchapfont}{\protect\bfseries}}
\addtocontents{toc}{\protect\renewcommand{\protect\cftchappagefont}{\protect\bfseries}}
\renewcommand{\cftchapleader}{\bfseries\cftdotfill{\cftchapdotsep}}
% ==========================================

\input{Bab I - Pendahuluan.tex}

\cleardoublepage % memastikan bab baru dimulai di halaman ganjil (kanan)
\chapter{STUDI LITERATUR}
\label{chap:studi-literatur}
\section{Tentang Studi Literatur}
Studi literatur biasanya berisi tentang tinjauan pustaka, landasan teori, dan penelitian-penelitian terdahulu yang relevan dengan topik tugas akhir yang dikerjakan.
Dalam subbab ini, penulis diharapkan dapat menjelaskan:
\begin{enumerate}
  \item Landasan teori yang diperoleh dari literatur yang akan digunakan untuk menyelesaikan persoalan TA;
  \item Pengetahuan tentang kasus yang akan dikaji; dan
  \item Penelitian atau solusi terkait untuk mengetahui posisi persoalan dan berbagai solusi yang memungkinkan untuk diterapkan berdasarkan hasil studi literatur ini.
\end{enumerate}
Jadi, studi literatur ini bukan sekadar rangkuman dari berbagai sumber referensi, tetapi harus diolah dan disajikan secara sistematis sehingga dapat memberikan gambaran yang jelas tentang landasan teori dan penelitian-penelitian terdahulu yang relevan dengan topik tugas akhir.

Penulisan studi literatur yang baik dimulai dari penjelasan tentang kasus yang akan dikaji, kemudian diikuti dengan penjelasan tentang teori-teori yang relevan, dan diakhiri dengan tinjauan terhadap penelitian-penelitian terdahulu yang berkaitan dengan topik tugas akhir.
Misalnya, jika topik tugas akhir adalah tentang masalah klasifikasi biji kopi berdasarkan citra kopi, 
penulis dapat memulai dengan menjelaskan terlebih dahulu tentang karakteristik biji kopi, masalah dalam melakukan klasifikasi biji kopi, 
kemudian menjelaskan tentang berbagai metode berbasis komputer yang dapat digunakan untuk mengklasifikasikan biji kopi berdasarkan citranya, 
metode terbaik yang memungkinkan untuk diterapkan, teori-teori dasar tentang metode tersebut, dan diakhiri dengan tinjauan terhadap penelitian-penelitian terdahulu yang menggunakan metode untuk klasifikasi citra.

Dalam studi literatur, biasanya banyak penjelasan tentang teori-teori yang disertai dengan rumus atau persamaan matematika, gambar, tabel, algoritma, pseudocode, atau kode program.
Oleh karena itu, pada subbab berikut akan dijelaskan beberapa hal teknis terkait format penulisan dokumen tugas akhir.

\section{Format Penulisan Gambar, Tabel, Rumus, dan Kode Program}
Berikut adalah beberapa contoh penulisan gambar, tabel, rumus, dan kode program yang sesuai dengan format penulisan tugas akhir.
\subsection{Penulisan Gambar}
Contoh gambar dapat dilihat pada Gambar \ref{fig:jaringan} dan Gambar \ref{fig:jaringan2}.
Secara vertikal, gambar biasanya diletakkan di bagian atas halaman (\textit{top}) atau di bagian bawah halaman (\textit{bottom}).
Gambar tidak harus diletakkan tepat di tempat gambar tersebut disebutkan pertama kali dalam teks,
tetapi dapat diletakkan di halaman berikutnya jika tidak muat di halaman yang sama. Pada contoh ini, kedua gambar diletakkan di bagian atas halaman 
menggunakan opsi \texttt{[t]} pada lingkungan \texttt{figure}, namun pada halaman yang berbeda, meskipun disebutkan di halaman yang sama akibat ruang yang terbatas. 

Penyebutan judul gambar dalam teks menggunakan perintah \texttt{ref}, misalnya Gambar \texttt{\\ref\{fig:jaringan\}}.
Gunakan huruf kapital pada kata "Gambar" saat menyebutkan gambar dalam teks.

\begin{figure}[t] % pilihan opsi yang disarankan: t = top, b = bottom, h = here
  \caption{Contoh peletakan gambar di bagian atas halaman} 
	\label{fig:jaringan}
    	\includegraphics[width=0.7\textwidth]{images/gambar1.png}
\end{figure}

Gambar, grafik, atau ilustrasi harus memiliki keterangan atau judul (\textit{caption}) yang menjelaskan isi gambar tersebut. 
Secara horisontal, gambar dan judulnya diposisikan di tengah. Nomor gambar tidak diakhiri tanda baca.
Judul gambar diletakkan di bawah gambar dan ditulis dengan huruf kecil kecuali huruf pertama pada kalimat judul. 
Kalimat judul yang panjang dapat ditulis dalam beberapa baris seperti pada Gambar \ref{fig:jaringan2}.

Ukuran gambar yang ditampilkan dapat diatur dengan mengubah nilai \textit{width} dalam sintaks \textit{includegraphics}. 
Gambar \ref{fig:jaringan} memiliki lebar 70\% dari lebar teks, sedangkan Gambar \ref{fig:jaringan2} memiliki lebar 90\% dari lebar teks.
 

\begin{figure}[t] 
  \caption{Contoh peletakan gambar dengan judul yang panjang sehingga ditulis dalam beberapa baris} 
	\label{fig:jaringan2}
    	\includegraphics[width=0.9\textwidth]{images/gambar1.png}
\end{figure}

Pastikan gambar yang digunakan memiliki resolusi yang cukup tinggi agar tidak pecah atau buram saat dicetak. 
Gambar umumnya tidak tajam dan tulisan tidak jelas atau kabur jika gambar tersebut:
\begin{enumerate}[a.]
  \item diperoleh dari hasil cropping pada suatu halaman buku atau situs web;
  \item hasil pembesaran gambar yang gambar aslinya sebenarnya berukuran kecil; atau
  \item disimpan dalam resolusi kecil
\end{enumerate}
Ketidakjelasan gambar ini dapat dilihat pada garis-garis diagram yang tidak tegas dan tulisan-tulisan dalam gambar yang tampak kabur dan kurang jelas terbaca.

Untuk mendapatkan gambar yang tidak kabur (\textit{blur}), langkah-langkah berikut dapat digunakan:
\begin{enumerate}[(a)]
\item Hindari \textit{screenshot}. Gambar yang didapat di suatu pustaka atau referensi sebaiknya digambar ulang (tidak perlu sama persis), misalnya menggunakan PowerPoint, Canva, Figma, draw.io, atau yang lainnya.
\item Jika diagram atau ilustrasi digambar menggunakan draw.io, saat gambar disimpan ke format PNG atau JPG (\textit{export as}), lakukan \textit{zoom} ke minimal 300\% (\textit{the default value is} 100\%). 
\item Jika diagram digambar dengan menggunakan \texttt{PowerPoint} atau aplikasi yang mengolah gambar vektor lainnya, gambar dapat langsung disimpan asalkan resolusinya tinggi.
\end{enumerate}

\subsection{Penulisan Tabel}
Format penulisan tabel hampir sama dengan format penulisan gambar. Bedanya adalah judul tabel diletakkan di atas tabel, bukan di bawah tabel seperti pada gambar.
Ada tabel yang bisa dimuat dalam satu halaman (tabel pendek) dan ada tabel panjang yang yang tidak muat dalam satu halaman. Kedua jenis tabel tersebut disusun dengan cara yang berbeda.
Jika tabel sangat panjang, gunakan paket \textit{longtable} untuk membuat tabel yang dapat terpotong ke halaman berikutnya.
Namun, jika tabel dapat dimuat dalam satu halaman, gunakan lingkungan \textit{table} biasa, tidak menggunakan \textit{longtable}.
Usahakan tabel dapat ditulis dalam satu halaman, tidak terpotong ke halaman berikutnya, jika memungkinkan.

\subsubsection{Penulisan Tabel Pendek}
Contoh tabel pendek dapat dilihat pada Tabel \ref{tab:tabel1}, \ref{tab:tabel2}, dan \ref{tab:tabel3}. 
Pada kedua table ini, tabel dan judulnya diposisikan di tengah secara horisontal dan 
judul tabel diletakkan di atas tabel. Secara vertikal, tabel sebaiknya diletakkan di bagian atas atau bawah halaman.

Penyebutan judul tabel dalam teks menggunakan perintah \texttt{ref}, misalnya Tabel \ref{tab:tabel1}.
Gunakan huruf kapital pada kata "Tabel" saat menyebutkan tabel dalam teks.

Tabel \ref{tab:tabel1} menggunakan paket \textit{tabularx} untuk mengatur lebar kolom secara fleksibel. 
Selain itu, tabel ini juga menggunakan \textit{threeparttable} untuk menambahkan catatan kaki pada tabel 
dan opsi \texttt{X} agar lebar kolom fleksibel menyesuaikan lebar tabel.
Catatan kaki pada tabel ini menggunakan perintah \texttt{tablenotes} yang disediakan oleh paket \textit{threeparttable}.
Catatan kaki pada tabel ini menggunakan perintah \texttt{\textbackslash footnotemark} dan \texttt{\textbackslash footnotetext} untuk menambahkan catatan kaki pada tabel.

Tabel \ref{tab:tabel2} menggunakan format standar \texttt{| l | c | r |}. 
Tabel \ref{tab:tabel3} menggunakan lebar kolom tetap (\textit{fixed width}) menggunakan \texttt{p\{width\}}.

\begin{table} 
  \centering
    \begin{threeparttable}[t] % threeparttable agar bisa menambahkan catatan kaki pada tabel
    \caption{Contoh tabel yang menggunakan \texttt{tabularx} untuk mengatur lebar kolom secara fleksibel. 
    Selain itu, tabel ini juga menggunakan \texttt{threeparttable} untuk menambahkan catatan kaki pada tabel 
    dan opsi \texttt{X} agar lebar kolom fleksibel menyesuaikan lebar tabel.}
    \label{tab:tabel1}
    \begin{tabularx}{0.6\textwidth}{| c | X| l | r @{}|} % @{} removes extra space at edges, X makes the column flexible width
	    \hline
	    No & Nama 	& Satuan 		& Harga \\
	    \hline
	    1 & Buku 	& Exemplar	& 25000 \\
	    2 & Komputer	& Unit		& 2500000 \\
	    3 & Pensil	& Buah		& 118900 \\
	    \hline
	  \end{tabularx} 
    \begin{tablenotes}
      \footnotesize
      \item \footnotemark[1]Tabel harga berdasarkan data tahun 2023.
      \item \footnotemark[2]Sumber: \url{https://www.example.com}
    \end{tablenotes}
  \end{threeparttable}
\end{table}




\begin{table}[t] % pilihan opsi yang disarankan: t = top, b = bottom, h = here
  \caption{Tabel yang menggunakan format standar \texttt{| l | c | r |}}
  \label{tab:tabel2}
	\begin{tabular}{ | l | c | r | }
	\hline
	Nama 	& Satuan 		& Harga \\
	\hline
	Buku 	& Exemplar	& 25000 \\
	Komputer	& Unit		& 2500000 \\
	Pensil	& Buah		& 118900 \\
	\hline
	\end{tabular}
\end{table}


\begin{table}[t] % pilihan opsi yang disarankan: t = top, b = bottom, h = here
  \centering
  \caption{Tabel yang lebar kolomnya dibuat tetap}
  \label{tbl:tabel3}
	\begin{tabular}{  p{0.2\textwidth} | p{0.2\textwidth} | p{0.2\textwidth}  }
	\hline
	\textbf{Nama} 	& \textbf{Satuan} 		& \textbf{Harga} \\
	\hline
	Buku 	& Exemplar	& 25000 \\
	\emph{Komputer}	& Unit		& 2500000 \\
	Pensil	& Buah		& 118900 \\
	\hline
	\end{tabular}
\end{table}

% -- Example of importing table from external file --
\subsubsection{Mengimpor Tabel dari Berkas Eksternal}

Untuk kerapian penulisan, tabel tidak harus ditulis di dokumen utama. Sebagai contoh, Tabel \ref{tab:tabel4} diimpor dari berkas eksternal \texttt{table/tabel1.tex} menggunakan perintah \texttt{input}. 
Dengan demikian, jika tabel tersebut perlu diubah, cukup mengubah pada berkas eksternal tersebut tanpa perlu mengubah pada berkas dokumen utama ini.

\input tables/tabel1.tex


% -- Example of long table --
\subsubsection{Tabel yang Sangat Panjang}
Jika tabel terlalu panjang sehingga tidak muat dalam satu halaman, gunakan paket 
\textit{longtable} untuk membuat tabel yang dapat terpotong ke halaman berikutnya, 
seperti pada Tabel \ref{tab:long}. 

\input tables/longtable1.tex

\subsection{Penulisan Rumus atau Persamaan Matematika Menggunakan {\LaTeX} dan Penomorannya}
Contoh rumus matematika dapat ditulis seperti pada Persamaan \ref{eq:contoh1} di bawah ini. 
Penomoran persamaan diletakkan di sebelah kanan, dan rumus ditulis dalam mode \textit{display math}.
\begin{equation}
E = mc^2
\label{eq:contoh1}
\end{equation}

Contoh lain penulisan rumus matematika yang lebih kompleks dapat ditulis seperti pada Persamaan \ref{eq:rumus2}.

\begin{align}
f(x) &= ax^2 + bx + c \\
f'(x) &= \frac{d}{dx}(ax^2 + bx + c) \notag \\ % tidak menampilkan nomor pada baris ini
      &= 2ax + b \label{eq:rumus2}
\end{align}

Jika rumus terlalu panjang untuk ditulis dalam satu baris, gunakan lingkungan \textit{multline} seperti pada Persamaan \ref{eq:rumus3} di bawah ini.
\begin{multline} 
y = a_0 + a_1x + a_2x^2 + a_3x^3 + a_4x^4 + a_5x^5 + a_6x^6 + a_7x^7 \\
    + a_8x^8 + a_9x^9 + a_{10}x^{10} \label{eq:rumus3}
\end{multline}

Jika ada penurunan rumus yang terdiri dari beberapa baris, namun tidak memerlukan penomoran pada setiap baris, gunakan lingkungan \textit{align*}, misalnya:

\begin{align*} 
S &= \sum_{i=1}^{n} i^2 \\
  &= 1^2 + 2^2 + 3^2 + \cdots + n^2 \\
  &= \frac{n(n + 1)(2n + 1)}{6}
\intertext{Contoh lainnya adalah rumus untuk mencari nilai rata-rata fungsi $f(x)$ pada interval $[p, q]$:}
\bar{f} &= \frac{1}{q - p} \int_{p}^{q} f(x) \, dx \\
        &= \frac{1}{q - p} \int_{p}^{q} (ax^2 + bx + c) \, dx \\
        &= \frac{1}{q - p} \left[ \frac{a}{3}x^3 + \frac{b}{2}x^2 + cx \right]_p^q \\
        &= \frac{a(q^3 - p^3)}{3(q - p)} + \frac{b(q^2 - p^2)}{2(q - p)} + c \label{eq:rumus4}
\end{align*}



\subsection{Penulisan Kode Program, \textit{Script} atau \textit{Pseudocode}}
Contoh penulisan kode program, \textit{script}, atau \textit{pseudocode} dapat dilihat seperti pada \textit{listing} \ref{lst:binary_iter}. 
Gunakan paket \textit{listings} untuk menulis \textit{source code} dalam bahasa pemrograman tertentu. 
Kode seperti ini disebut dengan istilah \textit{listing}. 
Kode program, \textit{script}, atau \textit{pseudocode} yang relatif pendek dapat ditulis di dalam
dokumen utama. Namun, jika kode programnya sangat panjang, sebaiknya ditulis di berkas terpisah dan diletakkan di lampiran.

\subsection{Penulisan Algoritma}
Contoh penulisan algoritma dapat dilihat pada Algoritma \ref{alg:binary-search}. Ada beberapa paket yang dapat digunakan untuk menulis algoritma,
misalnya \textit{algorithm2e}, \textit{algorithmic}, dan \textit{algpseudocode}. 
Pada contoh ini digunakan paket \textit{algorithmic} untuk menulis algoritma. Setiap paket memiliki sintaks yang berbeda-beda.
Oleh karena itu, pelajari dokumentasi paket yang digunakan untuk menulis algoritma.


\input{listings/binary-search-python.tex}


\input{algorithms/binary-search-alg.tex}





\input{Bab III - Analisis.tex}

% ==========================================
% BAB IV DESAIN KONSEP SOLUSI
% ==========================================
\chapter{PERANCANGAN}
\label{chap:perancangan}
Ilustrasikan desain konsep solusi dalam bentuk model konseptual dan penjelasan secara ringkas, 
beserta perbedaannya dengan sistem saat ini. Ilustrasi harus dapat dibandingkan (\textit{before} and \textit{after}). 
Karena masih berupa proposal, bab ini hanya berisi gambar desain konsep solusi tersebut dan 
penjelasan perbandingannya dengan gambar sistem yang ada saat ini (yang tergambar di awal Bab \ref{chap:analisis-masalah}).

% ==========================================
% BAB V RENCANA SELANJUTNYA
% ==========================================
\chapter{IMPLEMENTASI}
\label{chap:implementasi}
asdfasdf


% ==========================================
% BAB V RENCANA SELANJUTNYA
% ==========================================
\chapter{Evaluasi}
\label{chap:evaluasi}
asdfasdf


% ==========================================
% BAB V RENCANA SELANJUTNYA
% ==========================================
\chapter{PENUTUP}
\label{chap:penutup}
\section{Kesimpulan}
Jelaskan secara detail langkah-langkah rencana selanjutnya, hal-hal yang diperlukan atau akan disiapkan, dan risiko dan mitigasinya, yang meliputi:
\section{Saran}
asdfas




% ==========================================
% DAFTAR PUSTAKA
% ==========================================
\cleardoublepage
\printbibliography[title={DAFTAR PUSTAKA}]

% ==========================================
% LAMPIRAN (optional)
% ==========================================
\appendix

% Uncomment baris di bawah ini jika ada lampiran
\begin{appendices}
  %\makeatletter 
  %\renewcommand{\thechapter}{\@Alph\c@chapter} 
  %\makeatother
\cleardoublepage
\chapter{\textit{SOURCE CODE}}
\section{Perangkat Lunak untuk Akuisisi Data dari Sensor Ultrasonik}
\lstinputlisting[language=Python, caption=Binary search code]{listings/binary-search-python.tex}  
\section{Perangkat Lunak untuk Pengolahan Data Akuisisi dan Visualisasi Hasil Pengukuran Jarak} 
asdjfkjashdkjfas

\input{Lampiran-B.tex}
\end{appendices}
\backmatter

\end{document}
 